%!TEX program = xelatex
%!TEX TS-program = xelatex
\documentclass[12pt]{article}
\usepackage{color}
\usepackage{url}
\usepackage{fontspec,xltxtra,xunicode}
\usepackage{graphicx} 
\usepackage{amsmath}
\usepackage[colorlinks,linkcolor=red]{hyperref}

%--------------------中文设置--------------------
\defaultfontfeatures{Mapping=tex-text}
\setromanfont{Songti SC} %设置中文字体
\XeTeXlinebreaklocale “zh”
\XeTeXlinebreakskip = 0pt plus 1pt minus 0.1pt %文章内中文自动换行,可以自行调节

\newfontfamily{\H}{Heiti SC} %设定新的字体快捷命令
\newfontfamily{\E}{Weibei SC} %设定新的字体快捷命令
%--------------------中文设置--------------------

\title{ACM-ICPC 管理员配置 for ECUST}
\author{DNA049 10110057 math112 ECUST}

\begin{document}
\maketitle


\section{说点啥呢}

作为华理数学系本科生,有幸于大二结束的那个暑假加入母校ACM集训队。开始了我大学一段美好的计算机算法学习时光。2015年5月,母校举办ACM-ICPC上海邀请赛,2015年11月举办区域赛。作为ACMer有幸能负责整个比赛的软件系统配置,确实是一件幸福的事。特此记录



\section{Ubuntu的下载和安装}

Ubuntu镜像来自\href{http://mirrors.163.com}{网易开源镜像}。

下好Server版和Desktop版之后可以在windows下使用win32diskimager制作U盘启动盘(最后可用大白菜还原U盘)

Server版安装的一些细节:
\begin{enumerate}
\item
安装时插上网线,通过学校内网的DHCP初始化网络配置
\item
安装时会有选择安装包,其中Java一定要安装,其它几个随意
\item
安装版本最好与最近一年ACM-ICPC World Final一致
\item
为了比赛安全,服务器内存最少16G
\item
最好找一台服务器通过rsync做备份(虽然完全没必要)
\item
服务器任何时候都不需要连接外网的
\end{enumerate}

Desktop版正常安装即可。唯一需要注意的是,它需要发射到200多台机器上,所以样机一定要考虑周全再发射,因为发射是相当费时的。发射完之后调试好,所有机器没问题了之后马上备份,练习赛结束之后还原。

Desktop版安装的几点注意事项
\begin{enumerate}
\item
因为发射实际上是整个磁盘copy,所以可能有些不该copy的也copy了,一个直接的问题是被copy的机器会多出一个eth1。解决方法是在样机上执行 \\
rm /etc/udev/rules.d/70-persistent-net.rules
\item
发射之后如果不用dhcp来配置所有队伍机器的IP的话,每台机器设置IP还是很麻烦的。所以这是可以在样机当前目录~下建一个setip.sh。内容如下:
\begin{verbatim}
#!/bin/bash
netfile='/etc/network/interfaces'
echo $netfile
echo 'auto eth0' >> $netfile
echo 'iface eth0 inet static' >> $netfile
echo 'address 10.0.0.'$1 >> $netfile
echo 'netmask 255.255.255.0' >> $netfile
echo 'gateway 10.0.0.1' >> $netfile
\end{verbatim}
然后每台样机执行./setip id重启即可。
\item
一般比赛要禁掉U盘并开启防火墙的 \\
sudo mv /lib/modules/3.16.0-38-generic/kernel/drivers/usb/storage/usb-storage.ko /lib/modules/3.16.0-38-generic/kernel/drivers/usb/storage/usb-storage.ko.bak \\
sudo ufw enable \\
执行上述命令重启即可(原理是改驱动名,U盘驱动就无效了)。当然3.16.0-38-generic
可能需要相应修改(与uname -r下显示的一致即可)。
裁判方机器和管理员机器发射完后再自己设置回来即可。
\end{enumerate}
上述注意事项都是在发射前操作的。

\subsection*{IP分配}
比赛现场搭环境之前,肯定需要约定好服务器IP,打印服务器IP,榜单IP。以及所有客户端的IP的设置规则(拉网线,建局域网需要学校计算中心的帮助)。一般说来,队伍在220以内,所以C类网络就够用来。子网掩码可以设置成255.255.255.0。服务器留2个IP,以一个为主,打印服务器和榜单服务器类似。管理员和裁判的IP也可以预留,虽然不太必要,队伍端利用剩余的IP。例如:
可以让前20个ip分给服务器和裁判机器。当然裁判机器可以看作与队伍机器等价。这样前10个ip留出来就够了。比如10.0.0.10给管理员(最好把榜单和管理员放在一台机器上,这样做很方便,并且不要让管理员判题,防止崩溃)。9给服务器,8给备用服务器,7给打印机,至少要有两台机器配有打印机程序,只要用一台就可以了。2-6留用(1是网关,255是广播)。11-254可以分给客户端和裁判(如果是手动设置IP11-20就分给裁判比较好)。

sudo vim /etc/network/interfaces \\
auto lo \\
iface lo inet loopback \\
auto eth0 \\
iface eth0 inet static \\
address 10.0.0.240 \\
netmask 255.255.255.0 \\
gateway 10.0.0.1

设置完后 sudo ifdown eth0后再sudo ifup eth0

如果某些机器IP出了问题,需要修改IP,设置前可以先ping一下防止IP冲突。

\section{PC2的下载和配置}

PC2软件(管理员手册在doc目录下)可在\href{http://www.ecs.csus.edu/pc2/}{Programming Contest Control System}中找到PC2 system下载。作为管理员,必须将手册的每一个细节都搞清楚试清楚。


\subsection{服务器配置PC2}
在校内服务器连外网是相当麻烦的,其实也是没有必要的。我们将pc2-*.tar.gz下载到U盘,用U盘导入到服务器就行了。
\begin{enumerate}
\item
插入U盘前后调用cat /proc/partitions来查看此U盘的所在路径,一般是在sdb1(sdc1等)。
\item
调用mount -t -vfat /dev/sdb1 /mnt 将U盘挂载至/mnt目录下。
特别注意的是/mnt目录是不允许解压的,即使使用root权限也不行。
\item
cp /mnt/pc2*.tar.gz ~/ 将pc2复制到当前用户的目录下。\\
tar -zxvf pc2*.tar.gz  解压pc2
\item
为了方便起见,可以讲pc2的pc2v9.ini文件copy到其bin目录中(客户端也是这样)
\item
{\color{green}{特别注意:}} pc2sever文件一定要修改,最后一句中-Xms64M -Xmx1600M,为了使用更好的服务器性能,可以将Xms设置成服务器内存的1/16,Xmx设置成1/2。
\item
{\color{green}{运行:}} ./pc2server --nogui --login site1 --password site1 --contestpassword ecust520 打开pc2服务器。其中site1 site1是服务器默认的账号密码,ecust520是用户设置的密码,主要用于服务器关闭再次进入需要的密码以及,压缩包解压的密码。
\end{enumerate}
至此服务器的配置就完成了。


\subsection{客户端配置PC2}

下载或U盘导入PC2系统。

PC2的客户端有:1.管理员pc2admin,2.裁判pc2judge,3.队伍pc2team,4.榜单pc2board。

进入pc2将pc2v9.ini文件修改中localhost改成服务器的IP。

进入bin目录运行./pc2admin 进入登陆界面(请保证服务器开启)。账号密码默认为 root administrator1,比赛前务必修改root密码。这时你就可以产生账号了,例如10个judge,200个team,2个scoreboard。此后team和judge就可以登陆了。默认账号密码是一致的,例如team1,team1。其它的登录类似。

队伍端桌面有三个程序(快捷方式):打印,榜,pc2
这些程序都可以用c来写:
\begin{verbatim}
system("cd PC2HOME/bin&&PC2HOME/bin/pc2team");//pc2

system("firefox 10.0.0.7:8080"); // 打印

system("firefox 10.0.0.10:8060");//榜
\end{verbatim}
{\color{green}{特别:}}:解压pc2后一定要查看bin目录是否都是可运行的权限,否则进入 \\
cd [pc2]/bin  \\
chmod +777 * \\
来改变权限,如果在发射之后才发现就麻烦了。



\section{客户端软件安装}
再次说明,我们不可能每台机器都装一次Ubuntu,一般学校机房都是有装万欣恢复卡的。版本不同操作也不同,如何用万欣恢复卡装系统,以及如何系统备份,如何系统还原,如何发射,这些可以问校内计算中心的老师(一般F10设置各种东西包括发射和系统备份,Ctrl+r是还原)。安装好Ubuntu之后,默认是带有gcc和python的。连上外网后。\\
sudo apt-get update 更新源 \\
sudo apt-get upgrade 更新依赖库 \\
sudo apt-get intall g++ 安装g++ \\
openjdk-7-jdk,vim-gtk安装类似.

安装完编译器之后就开始安装编译器了,Codeblocks,Eclipse可以直接在ubuntu软件中心中安装,Eclipse版本会很低(不过无所谓,安装时务必记得勾选cdt就行了,因为部分队伍喜欢用Eclipse写c,c++)。
ubuntu软件中心安装的netbeans安装会导致无法使用,可以在官网安装稳定且最新的版本,安装十分简单,但是最好前面加sudo命令安装,这样安装的目录就不是在home目录下,而是在/etc等目录下。默认安装netbeans是不会在桌面显示图标的。可以到/usr/shar/applications中找到netbeans图标。将其拖到桌面或者启动栏。在pc2官网下载与服务器版本一致的pc2。注意netbeans可以不装。\\	
客户端的软件安装就告一段落来。



\section{最后的配置}
请确保在此之前,服务器和客户端的安装都以成功。现在选择2台客户端作为打印服务器,榜单服务器。下面分别介绍其安装


\subsection{打印机安装}
购买3-5台hp打印机,Ubuntu是默认有hp打印程序。将打印机全部插上,在打印服务器(可以用dpkg -l hplip查看打印程序内核,用hp-setup安装打印机驱动。
安装好之后在设置打印池,打印机设置add中有两个选项,其中一个就是clas,设置一下就行了,并将这个class设为默认打印机。利用lp filename看是否离线打印。

下面介绍一下网络打印。ccpc提供了该网络打印的\href{http://board.acmicpc.cn/soft.html}{程序}。使用安装如下:
\begin{enumerate}
\item
Install CUPS and Mongo DB \\
sudo apt-get install cups-pdf \\
sudo apt-get install mongodb
\item
在/tmp目录下新建个目录叫srcfiles
\item
unzip prserv1.6.zip
\item
sudo MONGODB=WARMUP prserv-1.6c/bin/prserv -Dhttp.port=8080
\item
访问 http://[prservip]:8080  \\
Default admin passowrd is ccpc2015cn \\
account and password file format: \\
team1[TAB]password[TAB]seatno 
\end{enumerate}


\subsection{榜单设置}
榜单,在pc2/bin/html下会默认生成一些榜(但无index.html),为此我们可以将pc2-*/samps/web/xsl/wf.index.xsl删掉带png的一行并且更名为index.html放置在pc2-*/data/xsl目录下就可以使用index.html作为队伍们可以看到的榜。在/pc2/bin/html文件下调用python -m SimpleHTTPServer 8060(大于6000且与客户端所配置的一致就行,随意),其它地方用火狐输入榜单ip:8060就可以看到榜了。

比赛时两个榜单,一个是公共榜单scoreboard1(最后一小时关闭)在打印服务器上开启,一个是用于送气球的榜单scoreboard2可放在管理员机器上(节约机器),需要在管理员端进行配置。

\subsection{气球设置}
生成气球的话pc2是在admin这边设置的,notification这边选择site和scoreboard客户端号,配置题目对应颜色和loopprint路径即可,具体如下
假设scoreboard1用于榜单,scoreboard2用于气球生成,那么在管理员端的notification这边设置选择scoreboard2。print device设置为/home/username/ballonprinter(其实随便什么都行)。然后在装打印机的机器上,开启打印服务。localhost里面在此设置一下class。设置好之后,将pc2提供的loopprint文件进行修改。进入localhost:631为printer设置class。路径设置成/home/username/ballonprinter(与上面一致)。执行./loopprint即可。可以点击测试打印查看。


\subsection{管理员比赛配置}
pc2服务器开启后,pc2admin就应该:
\begin{enumerate}
\item
设置账号(先产生帐号,后导入文件设置帐号密码)
\item
设置语言(C,C++,java)和相应参数
\item
添加题目,special judge可看文档和例子题目对应的气球以及气球设置,比赛名设置,裁判可以看到的信息设置,默认回复设置,auto judge设置,开始比赛。
PC2的使用还有很多细节都可以在官方文档中查到,作为管理员有责任和义务通读掌握。
\item
推荐使用稳定版本pc2-9.2.3。
\item
管理员不要判题。练习赛是很多队伍都会提交各种坑爹代码,所以管理员不要判题,正常情况也是没有时间判题的。
\item
最好不要将客户端机器的密码告诉参赛人员。
\item
如果出现pc2无法连接或闪退情况,第一件事是检测队伍段机器网线接口是否正常。如果出现pc2死掉,ctrl+alt+T进入终端killall java即可。如果出现死机,可以选择ctrl+alt+F1后ctrl+alt+del快速重启机器,对队伍没有太大的影响。
\item
所有服务器必须用千兆网线,客户端和裁判端可以用百兆网线。另外榜单服务器,最好使用nginx配置,因为SimpleHTTPServer是单线程,可能会被某台机器卡死。

\end{enumerate}


\section{写在最后}

比赛开始后,基本管理员和裁判前10分钟会比较闲,之后就开始忙碌了,注意每一个autojudge至少要保证一个没崩溃,不然就有些麻烦了,记得1小时前封榜(只关一个榜,管气球的榜不要关了。)。最后的10分钟超级忙碌,不过也不要慌乱,判题有延迟也无所谓的,只要不判错就ok。判完题后可以开启关闭的榜,进行更新榜单。至于滚榜这一点,我们没有做,实际上ccpc也做了这个程序,只是当时木有拿来。(很感谢周专家的帮忙和几位学长,老师和志愿者的帮助)。

希望该文不仅仅作为纪录,还能给你提供一些微薄的帮助。


\end{document}