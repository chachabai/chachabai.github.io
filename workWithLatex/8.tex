%!TEX program = xelatex
%!TEX TS-program = xelatex
\documentclass[12pt]{article}
\usepackage{url}
\usepackage{fontspec,xltxtra,xunicode}
\usepackage{graphicx} 
\usepackage{amsmath,amsfonts} 

\defaultfontfeatures{Mapping=tex-text}
\setromanfont{Songti SC} %设置中文字体
\XeTeXlinebreaklocale “zh”
\XeTeXlinebreakskip = 0pt plus 1pt minus 0.1pt %文章内中文自动换行,可以自行调节

\newfontfamily{\H}{Heiti SC} %设定新的字体快捷命令
\newfontfamily{\E}{Weibei SC} %设定新的字体快捷命令

\author{ 15110840001 陈智鹏}
\title{ 仅在一点连续的函数例子 }

\begin{document}

\maketitle
\E{
在一元微积分中,有一个广为人知的结论:一元函数在一点可导,必在该点连续,即可导必连续。

自然会有这样一个问题:

	一元函数在一点可导能否推出它在该点的一个小邻域连续呢?

这个想法是很自然的,不严格的思考可能会认为应该是对的,但是它并不成立。下面给出一个反例:
\[ f(x) = x^2 D(x) = \left\{ \begin{array}{ll} 0 & x \in \mathbb{Q} \\ x^2 & x \notin \mathbb{Q} \end{array} \right. \]
其中$D(x)$ 为Dirichlet函数。

容易验证函数$f(x)$在$x=0$处可导,但在$x \neq 0$处不连续。即否定了上述问题。

最后,类似地,我们还可以通过Dirichlet函数构造$\mathbb{R}$上一些仅在有限个点连续的函数。也可以通过周期函数构造仅在所有整数点连续的函数。但是由Baire纲定理可以证明,不存在在所有有理数点连续,无理点间断的函数。最后Riemann函数给出了一个在所有有理数点间断,无理点连续的函数。这些反例使得人们对函数连续的概念有了更感性的认识。

}
\end{document}