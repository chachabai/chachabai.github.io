\documentclass[12pt]{amsart}
\usepackage{amsmath,amssymb,amsbsy,amsthm}
\usepackage[hmargin=1.4in,vmargin=1.4in]{geometry}
\usepackage{url}

\usepackage{tikz}
%\usetikzlibrary{matrix}
% hyperlinks
\usepackage[linktocpage]{hyperref}
%\usepackage[linktocpage]{hyperref}
\hypersetup{
    colorlinks=true,        % false: boxed links; true: colored links
    linkcolor=red,          % color of internal links (change box color with linkbordercolor)
    citecolor=red,         % color of links to bibliography
    filecolor=magenta,      % color of file links
    urlcolor=cyan           % color of external links
}

%Chapter/heading format
% \usepackage{titlesec}
% \titleformat{\section}[block]
%   {\filcenter}{\thesection .}{0.5em}{}
% \titleformat{\subsection}[runin]
%   {\normalfont}{\bfseries \thesubsection}{0.5em}{\bfseries}
% \titlespacing*{\subsection}{0pt}{22pt}{0.5em}

% This removes subsections from tableofcontents
\addtocontents{toc}{\protect\setcounter{tocdepth}{1}}

% Changing \mathcal{C}
\usepackage{eucal}

\newcommand{\red}{\color{red}}
\newcommand{\blue}{\color{blue}}

% some useful commands
\newcommand{\C}{ \mathbb{C} }
\newcommand{\R}{ \mathbb{R} }
\newcommand{\Q}{ \mathbb{Q} }
\newcommand{\Z}{ \mathbb{Z} }
\newcommand{\N}{ \mathbb{N} }
\newcommand{\F}{ \mathbb{F} }
\newcommand{\A}{ \mathbb{A} }
\DeclareMathOperator{\GL}{GL}
\newcommand{\ngr}{\trianglerighteq}
\newcommand{\ngl}{\trianglelefteq}
\newcommand{\ngrneq}{\triangleright}
\newcommand{\nglneq}{\triangleleft}
\newcommand{\cl}[1]{ \overline{#1} }

\newcommand{\aA}{ \mathcal{A} }
\newcommand{\bB}{ \mathcal{B} }
\newcommand{\cC}{ \mathcal{C} }
\newcommand{\dD}{ \mathcal{D} }
\newcommand{\fF}{ \mathcal{F} }

\newcommand{\Prim}{ \text{Prim } }

% theorem/proof environments, numbering, etc
%\newtheoremstyle{slstyle}% name of the style to be used
%  {}% measure of space to leave above the theorem. E.g.: 3pt
%  {}% measure of space to leave below the theorem. E.g.: 3pt
%  {\slshape}% name of font to use in the body of the theorem
%  {0pt}% measure of space to indent
%  {\bfseries}% name of head font
%  {. }% punctuation between head and body
%  {1pt}% space after theorem head; " " = normal interword space
%  {}% Manually specify head

%\swapnumbers

\theoremstyle{plain}
	\newtheorem{thm}{Theorem}
	\newtheorem{notn}[thm]{Notation}
		\numberwithin{thm}{section}
	\newtheorem{lemma}[thm]{Lemma}
	\newtheorem{prop}[thm]{Proposition}
	\newtheorem{cor}[thm]{Corollary}
	\newtheorem{conj}[thm]{Conjecture}
    \newtheorem{prob}[thm]{Problem} 

	\newtheorem*{thm*}{Theorem}
	\newtheorem*{lemma*}{Lemma}
	\newtheorem*{prop*}{Proposition}
	\newtheorem*{cor*}{Corollary}
	\newtheorem*{conj*}{Conjecture}

\theoremstyle{definition}
	\newtheorem{example}[thm]{Example}
	\newtheorem*{example*}{Example}
	\newtheorem{defn}[thm]{Definition}
	\newtheorem{remark}[thm]{Remark}

\title{Primitive Ideal}
\author{Chen Zhipeng}
\usepackage{amsfonts}
\begin{document}
\maketitle

\section*{Baire Space}

\defn A topology space is called Baire if $\cap_{i \in I} O_i$ is dense, where $O_i$ is dense open set, and $|I| < |k|$.

\remark If $R$ is countable generated, we may repalce all $ < |k|$ by countable.

\lemma Let $R$ be a prime semiprimitive ring. $I$ is a prime ideal in $R$, $I = 0$ if and only if $V(I) = Prim R$.

If $V(I) = Prim R$, $I = \cap_{Q \text{ is primitive}} Q = 0$.

\lemma Let $R$ be a prime semiprimitive ring. Then every non-empty open set is dense.

\begin{proof}
Suppose $U$ is a non-empty open set, $V$ is open and $U \cap V = \emptyset$. Let $I = \cap_{P \notin U} P$, $J = \cap_{P \notin V} P$, then $IJ \subseteq I \cap J = 0$, but $U \neq \emptyset$, so $I \neq 0$, so $J = 0$. So $V = \emptyset$. 
\end{proof} 

\section*{Kaplansky Ring}
\defn $R$ is said to be Kaplansky ring provided the primtive ideal space of every homomorphic image of $R$ is a Baire space.

\lemma Suppose $R$ is Jacobson ring which is (one-side) noetherian, Then $R$ is Kaplansky if and only if $\Prim R/P$ is Baire for every prime ideal $P$ in $R$.

Note that $\Prim R/I = (\Prim R/P_1) \cup \cdots \cup (\Prim R/P_n)$, where $\{P_1,\cdots, P_n\}$ is the set of prime ideal mininal over $I$. and the proof is clear.

\section*{Primitive Spectrum}

$\Prim R$ is sub-topological space of $Spec(R)$ consist of (left) primitive ideal.

\lemma Let $R$ be a prime semiprimitive ring. $\Prim R$ is Baire if and only if $\cap_{i \in I} U_i \neq \emptyset$ for every $|I| < |k|$ and $U_i, i \in I$ are non-empty open set in $\Prim R$.

\begin{proof}
$U \doteq \cap_{i \in I} U_i$, for every non-empty open set $V$, $U \cap V = \cap_{i \in I}  {U_i} \cap V$ is non-empty by assumption. So $U$ is dense.
\end{proof}

We may write above lemma as:

\lemma Let $R$ be a prime semiprimitive ring. $\Prim R$ is Baire if and only if $\cup_{i \in I} W_i$ is proper for every $|I| < |k|$ and $W_i, i \in I$ are proper closed set in $\Prim R$.

thus we may sate:

\lemma Let $R$ be a prime semiprimitive ring. $\Prim R$ is Baire if and only if for every set of ideal ${J_i}_{i \in I}$ with $|I| < |k|$, there exist an (left) primitive ideal don't contain any $J_i$.

If $R$ is (left) noetherian, $R$ has only finitely many prime ideal minimal over $J_i$. So we have

\lemma Let $R$ be a noetherian prime semiprimitive ring. $\Prim R$ is Baire if and only if for every set of prime ideal ${P_i}_{i \in I}$ with $|I| < |k|$, there exist an (left) primitive ideal don't contain any $P_i$.

Assume that $k$ is a field, $R$ is a Noetherian $k$-algebra, with $\dim_k(R) < |k|$, so $R$ is a Jacobson ring, satisfies Nullstellensatz, thus if $P$ is a primitive, then it is rational, if $P$ is locally closed, then $P$ is primitive.

In this case, J.P.Bell have $0$ is Rational if and only if there is a set $X$ of cardinality less than $|k|$ and a set of nonzero prime ideals $\{P_x: x \in X\}$ such that every nonzero prime ideal $P$ of $R$ contains $P_x$ for some $x \in X$.

So If $R$ is prime ring satisfies above assumption. Then $0$ is primitive if and only if $0$ is rational and $\Prim R$ is Baire.

If $P$ is a Prime ideal in $R$, Then $P$ is primitive if and only if $P$ is rational and $\Prim R/P$ is Baire.

\remark $\Prim R$ is Baire in above case has easy version.


\begin{thebibliography}{2}  
\bibitem{ref1} The Dixmier-Moeglin equivalence, Morita equivalence, and homeomorphism of spectra. (English summary)
J. Algebra 534 (2019), 228–244.
\bibitem{ref2} Baire category and Laurent extensions.
Canadian J. Math. 31 (1979), no. 4, 824–830.
\end{thebibliography}
\
\end{document}